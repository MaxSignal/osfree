%
% Manual for RegMath
% Copyright 2001, Patrick TJ McPhee
% Distributed under the terms of the Mozilla Public Licence
% You can obtain a copy of the licence at http://www.mozilla.org/MPL
% The Original Code is regmath
% The Initial Developer is Patrick TJ McPhee
% $Header: C:/ptjm/rexx/math/tmp/RCS/regmath.tex 1.2 2001/11/04 23:10:00 ptjm Exp $
%


\documentclass{article}

% if run through pdftex, use ps fonts and turn on hyperlinking
\ifx\pdfoutput\undefined
 \def\rarrow{\ensuremath{\to}}
 \def\rxpicture#1{<picture omitted>}
\else
 \usepackage[pdftex,pdfborder=0 0 0]{hyperref}

 \usepackage{mathptm}

 % redefine some symbols to avoid embedding CM fonts
 \font\bsfont=ptmr8r
 \def\textbackslash{{\bsfont\char`\\}}
 \font\symbol=psyr
 \def\rarrow{{\symbol\char174}}

 \pdfinfo { /Title (The Regina Rexx Interpreter -- Rexx Math Functions)
  /Author (Patrick TJ McPhee)
  /Subject (rxmath/amath User Guide)
  /Keywords (rexx,regina,rxmath,amath)
  /Copyright (Copyright � 2001, Patrick TJ McPhee)
 }
 \def\rxpicture#1{\begin{figure}[ht]\pdfximage{#1}\pdfrefximage\pdflastximage\end{figure}}
\fi

\def\bs{\textbackslash\penalty\hyphenpenalty}
\def\_{\textunderscore\penalty\hyphenpenalty}
\advance\textwidth by 1 in
\advance\textheight by 1 in
\advance\oddsidemargin by -.5in
\advance\topmargin by -.5in
\advance\labelwidth by 4 in

\usepackage{makeidx}

\makeindex

\begin{document}
\let\thepage\relax

\title{The Regina Rexx Interpreter -- Math Functions}
\author{Patrick TJ McPhee (ptjm@interlog.com)\\
        DataMirror Corporation}
\date{version 1.0.0, 27 October 2001}

\maketitle

\setcounter{page}{0}
\eject
\pagenumbering{roman}

\tableofcontents

\vfil\eject
\setcounter{page}{1}
\pagenumbering{arabic}

\section{Introduction}

This paper describes two function libraries which provide common
trigonometric functions to rexx programs.
Amath was written during the fall of 1998 in response to the newsgroup
posting contained in the file amigamath.txt. It is my hope that this
library is compatible with the amiga rexx math library. Rxmath was
written during the the fall of 2001 to provide a library compatible with
IBM's rxmath library, which had appeared earlier in the year.

The libraries are little more than wrappers around the C math
library---in particular they do not allow
arbitrary-precision\index{precision!arbitrary}
operations.
If arbitrary-precision calculations are required, I recommend John
Brock's\index{Brock!John} rxxmath\index{rxxmath} library, which is
written in rexx, and which does provide
arbitrary-precision support, at the expense of speed.

The two libraries are functionally equivalent, with these exceptions.
\begin{itemize}
\item RxMath
supports arguments in radians, degrees, or gradians, while amath
accepts radians only;
\item RxMath has a function which returns the value of $\pi$ (but $\pi =
3.141592653897832\dots$);
\item amath supports inverse hyperbolic functions on some (read: Unix) platforms;
\item amath provides secant and cosecant functions (but these are easily derived
from sine and cosine);
\item amath provides ceil and floor functions (but these can be achieved using
format(\thinspace)).
\end{itemize}

The functions are provided in hopes that they will be useful, but there
is no warranty.

\section{Housekeeping Functions}

\subsection{MathLoadFuncs/MathDropFuncs}

\index{Math!LoadFuncs}\index{Math!DropFuncs}In both libraries, the only exported
function is called MathLoadFuncs. It must be loaded using RxFuncAdd\index{RxFuncAdd},
then called to make the other functions available.
It was my intention for MathLoadFuncs to query the rexx interpreter for the
setting of numeric digits\index{precision!numeric digits}, ignoring subsequent changes
to the setting, however there is no way to do this with the current
Regina implementation, so it always ignores the numeric digits setting.

To make the RxMath library functions available:

\begin{verbatim}
  call rxfuncadd 'mathloadfuncs', 'rxmath', 'mathloadfuncs'
  call mathLoadFuncs
\end{verbatim}

while to make the amath library functions available:

\begin{verbatim}
  call rxfuncadd 'mathloadfuncs', 'rexxmath', 'mathloadfuncs'
  call mathLoadFuncs
\end{verbatim}

It is possible to load both libraries in the same program, however the
statements above reuse the function name `mathloadfuncs'. Since you
only ever need to call the function once for each library, this doesn't
matter, but if it bothers you, the third argument to rxFuncAdd is the
name of the function within the rexx interpreter, and you can change it
to anything you like.

If you have an error loading these libraries\index{troubleshooting}
(or any library), you can
sometimes get useful information by calling the Regina-specific function
rxfuncerrmsg:

\begin{verbatim}
  if rxfuncadd('mathloadfuncs', 'rexxmath', 'mathloadfuncs') then
    say rxfuncerrmsg()
  else
    call mathLoadFuncs
\end{verbatim}

I know of a few problems which will prevent these libraries from loading:
If you use rexx.exe\index{rexx.exe} from the Regina distribution,
you cannot load external function libraries. You must use regina.exe\index{regina.exe}
instead. The difference between the two programs is that rexx.exe has
the Rexx interpreter linked into the executable, while regina.exe loads
it from a shared library. Programs run faster in the statically linked
interpreter, but it can't load function libraries (it's not technically
possible on Windows, although it is on most Unix systems, Regina still doesn't
allow it).

If you are using the initial release of Windows 95, it does not
include the file msvcrt.dll\index{msvcrt.dll}. You must get this file
from Microsoft support. Many applications install it, but I find that
unbelievable, since it's supposed to be part of the operating system.

Most systems expect the first argument to rxfuncadd to exactly match the
case of the function name in the shared library (it's all lower-case).
Unix systems expect the second argument to match the case of the file
name containing the shared object (which is generally all lower-case).
More recent versions of Regina have a work-around to allow the case in
the rxfuncadd call to differ from the actual case in the library.

\begin{verbatim}
  call mathDropFuncs
\end{verbatim}

After calling mathLoadFuncs, you can call mathDropFuncs to unregister all the
functions. This is a useful operation with IBM interpreters, since they make
function libraries global to all rexx programs, and there's no way to replace
a library without unloading it first. There's no real value when using Regina.

\section{RxMath Library}

The RxMath Library is a straight copy of the interface of IBM's RxMath
library, which appeared in March 2001. The implementation is based on IBM's
documentation, while this manual is based on my reading of the C code about a
month after implementation. It's shorter than IBM's manual, but I feel it's
more informative, and it's got some slick diagrams.

I expect that this library is fully compatible with IBM's library, with two
exceptions: First, I do no error checking, so non-numeric arguments are
silently converted to 0, while invalid arguments will generally result in a
return code of NaN, rather than ERROR. Second, I never check the numeric
digits\index{precision!numeric digits} setting, meaning a default precision of
16 is in effect always. Most functions take precision as an argument, so lower
precisions can be achieved either by adding 0 to the result, or by passing the
desired precision to the function. I believe that the differences will result
in a measurable improvement in performance.

The function naming convention is `RxCalc' followed by a common abbreviation
of the trigonometric function. Thus `sine' becomes `RxCalcSin'.

\subsection{Trigonometric Functions}

Trigonometric functions are functions on an angle\label{sec:trig}, $\theta$, which
we typically define in terms of the cartesian co-ordinates, $(x,y)$,
of some point $(r,\theta)$. $x$, $y$, and $r$ can be taken as the
sides of a right-angled triangle, as shown in the figure.

\rxpicture{rat.pdf}

Because of the triangle thing, trigonometric functions are often used in
geometrical applications. Because they have a boring periodic nature,
they are also used in used in signal processing applications. For
instance, sine waves are used at the start of `Lemmings' in the classic
album {\it Pawn Hearts}\index{Vander Graaf Generator}.

\rxpicture{sincos.pdf}

In mathematics, angles are typically measured in radians\index{radian}.
The size of the
angle in radians is the ratio between the arc of a circle which subtends the
angle and the radius of the circle. The traditional, non-mathematical
measure of angles is in degrees\index{degree}, minutes, and seconds, which are a
base-60 system, and so presumably go back to the Phonecians. There are
360 degrees in a circle, 60 minutes in a degree, and 60 seconds in a
minute, althought you don't need to know that, since the library takes
degrees as decimal numbers. Finally, engineering calculations often use
gradians\index{gradian}, which divide the circle into 400. The rxmath library accepts
all three measures and can return all three types.

Since calculations involving radians typically involve $\pi$, there's a
function which returns it (see below for a description of the {\it precision}
argument:
\begin{quote}
RxCalcPi\index{RxCalc!Pi}({\it precision})
\end{quote}

The calling sequences of each trigonometric function is the same, so
I'll describe one, then give a list of all of them.

\begin{quote}
RxCalcSin({\it value}[, {\it type}][, {\it precision}]) {\rarrow} {\it
result}
\end{quote}

RxCalcSin\index{RxCalc!Sin} takes an angle ({\it value}) and returns its
sine.
If {\it type} is not specified or begins with `d', {\it value} is
taken to be in degrees; if {\it type} begins with `r', {\it value} is taken
to be in radians; and if {\it type} begins with `g', {\it value} is taken to
be in gradians. For the inverse functions, {\it type} determines the
type of the return value.

{\it precision} over-rides\index{precision!setting} the current setting of
numeric digits\index{precision!numeric digits}. It must be a whole number
between 1 and 16, inclusive. Note that in this implementation, numeric digits
is intended to be checked once at the time MathLoadFuncs\index{Math!LoadFuncs}
is called (but is in fact never checked, since the SAA API doesn't provide a
way to do it).

\begin{tabular}{lp{10cm}}
RxCalcSin&{\it value} is an angle, and the function returns its sine;\\
RxCalcArcSin\index{RxCalc!ArcSin}&{\it value} is a number in the range $-1,1$,
and the function returns the angle in the range $-{\pi \over 2},{\pi \over
2}$ whose sine is {\it value} (the inverse sine);\\
RxCalcCos\index{RxCalc!Cos}&{\it value} is an angle, and the function
returns its cosine;\\
RxCalcArcCos\index{RxCalc!ArcCos}&{\it value} is a number in the range $-1,1$
and the function returns its inverse cosine in the range $0,\pi$;\\
RxCalcTan\index{RxCalc!Tan}&{\it value} is a number which is not a multiple
of $\pi$, and the function returns its tangent;\\
RxCalcArcTan\index{RxCalc!ArcTan}&{\it value} is a number, and the
function returns its inverse tangent in the range $-{\pi\over 2},{\pi \over 2}$;\\
RxCalcCotan\index{RxCalc!Cotan}&{\it value} is a number which is not a
multiple of $\pi\over 2$, and the function returns its cotangent.\\
\end{tabular}

\subsection{Exponential Functions}

\label{sec:exp}The natural logarithm\index{logarithm!natural} is the
function $\log x = \int_1^x {1\over t}dt,\, x > 0$. It is the
anti-derivative of $1 \over x$. The natural
exponential\index{exponential!natural} function is its inverse function.
It turns out that the natural exponential function is $e^x$\index{Euler,
Leonhard}, where $e = \lim_{h\to 0}(1+h)^{1\over h}$. The exciting thing
about the natural exponential function is that it is its own derivative.

Most of the exponential and logarithmic functions follow this calling
sequence:

\begin{quote}
RxCalcExp({\it value}[, {\it precision}]) {\rarrow} {\it result}
\end{quote}

RxCalcExp\index{RxCalc!Exp} takes {\it value}, a real number, and
returns the natural exponent of the number.
As with the trigonometric functions, {\it precision} can be used to
override the numeric digits setting.

\begin{tabular}{lp{10cm}}
RxCalcExp&{\it value} is a real number, and the function returns its
natural exponent;\\
RxCalcLog\index{RxCalc!Log}&{\it value} is a positive real number, and the function
returns its natural logaritm;\\
RxCalcLog10\index{RxCalc!Log10}&{\it value} is a positive real number, and the function
returns its base-10 logaritm\index{logarithm!base 10}. Logarithms for other bases\index{logarithm!other bases} can be obtained by dividing by
10 and multiplying by the other number. {\it e.g.}, log2({\it value})
is given by .2 * RxCalcLog10({\it value});\\
RxCalcSqrt\index{RxCalc!Sqrt}&{\it value} is a positive real number, and
the function returns its positive square root.\\
\end{tabular}

\begin{quote}
RxCalcPower({\it base}, {\it exponent}[, {\it precision}]) {\rarrow} {\it result}
\end{quote}

RxCalcPower\index{RxCalc!Power} returns {\it base} raised to the {\it
exponent}th power. The standard Rexx exponential operator ** accepts only
whole-number exponents, but RxCalcPower accepts any real number for both the base
and the exponent.

\rxpicture{exp.pdf}

\subsection{Hyperbolic Functions}

\label{sec:hyper}Hyperbolic sine and cosine are the functions $\sinh x = {e^x -
e^{-x}\over 2}$ and $\cosh x = {e^x + e^{-x} \over 2}$. The reason for
the names is that $(\cos t, \sin t), 0 \le t \le 2\pi$ defines the
circle $x^2 + y^2 = 1$, while $\cosh t, \sinh t), t \in {\bf R}$ defines
one branch of the hyperbola $x^2 - y^2 = 1$. Hyperbolic cosine is used
in calculating things like sag and motion through resistive media. I
don't know what the other hyperbolic functions are good for, but they're
defined similarly to the trigonometric functions.

\begin{quote}
RxCalcSinH({\it value}[, {\it precision}]) {\rarrow} {\it result}

RxCalcCosH({\it value}[, {\it precision}]) {\rarrow} {\it result}

RxCalcTanH({\it value}[, {\it precision}]) {\rarrow} {\it result}
\end{quote}

RxCalcSinH returns the hyperbolic sine of {\it value} to {\it precision}
digits. Similarly, RxCalcCosH returns the hyperbolic cosine, and
RxCalcTanH returns the hyperbolic tangent ($\sinh x \over \cosh x$).

\section{AMath Library}

As I mentioned in the introduction, the amiga math library is based on
an almost anonymous newsgroup posting (reproduced in amigamath.txt), which
describes a math library for Amiga Rexx. The function names were
generally taken from the C math library, which probably got them from a
ForTran math library. This is by way of saying that there isn't a naming
convention.

I don't know if they are compatible with the Amiga library, but they've
been on offer for a few years without any bug reports, so they can't be
all bad [in testing the library for this release, I discovered that a
few functions simply couldn't be called, so I guess it's not that great
after all].

\subsection{Trigonometric Functions}

Please see section \ref{sec:trig} for a general discussion of the
trigonometric functions. The arguments to the amath trigonometric
functions and the return codes of the inverse trigonometric functions
are all in radians.
Most of them follow this calling sequence:

\begin{quote}
Sin({\it value}) {\rarrow} {\it result}
\end{quote}

\begin{tabular}{lp{10cm}}
Sin\index{Sin}&{\it value} is an angle, and the function returns its
sine;\\
ASin\index{ASin}&{\it value} is a number in the range $-1,1$,
and the function returns the angle in the range $-{\pi \over 2},{\pi \over
2}$ whose sine is {\it value} (the inverse sine);\\
Cos\index{Cos}&{\it value} is an angle, and the function
returns its cosine;\\
ACos\index{ACos}&{\it value} is a number in the range $-1,1$
and the function returns its inverse cosine in the range $0,\pi$;\\
Tan\index{Tan}&{\it value} is a number which is not a multiple
of $\pi$, and the function returns its tangent;\\
CoT\index{CoT}&{\it value} is a number which is not a
multiple of $\pi\over 2$, and the function returns its cotangent.\\
CoTan\index{CoTan}&synonym for cot\\
CSc\index{CSc}&{\it value} is an angle, and the function
returns its cosecant;\\
Sec\index{Sec}&{\it value} is an angle, and the function
returns its secant;\\
\end{tabular}

The arctangent function has a slightly different syntax:

\begin{quote}
ATan\index{ATan}({\it value}[,{\it othervalue}) {\rarrow} {\it result}
\end{quote}

If {\it othervalue} is not specified, atan returns the inverse tangent
of {\it value} in the range $-{\pi\over 2},{\pi \over 2}$. If {\it
othervalue} is specified, atan returns the inverse tangent of $\it value
\over othervalue$ in the range indicated by the signs of the arguments.
If {\it value} is positive and {\it othervalue} is negative, ${\pi \over
2} \le {\it result} \le \pi$.

\subsection{Exponential Functions}

Please see \ref{sec:exp} for an interesting and informative disquisition
on the origins of the exponential functions.\footnote{But look no further
than this section to find the word `disquisition', possibly making its
first appearance in any software manual. Professional software companies
actually pay people to remove words like that from their manuals. I
can't afford to do that.}

The exponential functions have the same prototype as all the other
functions:

\begin{quote}
Exp\index{Exp}({\it value}) {\rarrow} {\it result}
\end{quote}

\begin{tabular}{lp{10cm}}
Exp&{\it value} is a real number and the function returns its
natural exponent;\\
Log\index{Log}&{\it value} is a positive real number, and the function
returns its natural logaritm;\\
Log10\index{Log10}&{\it value} is a positive real number, and the function
returns its base-10 logaritm. Logarithms for other bases can be obtained by dividing by
10 and multiplying by the other number. {\it e.g.}, log2({\it value})
is given by .2 * RxCalcLog10({\it value});\\
Sqrt\index{Sqrt}&{\it value} is a positive real number, and
the function returns its positive square root.\\
\end{tabular}

\begin{quote}
Pow\index{Pow}({\it base}, {\it exponent}) {\rarrow} {\it result} 
\end{quote}

Pow returns {\it base} raised to the {\it
exponent}th power. The standard Rexx exponential operator ** accepts only
whole-number exponents, but pow accepts any real number for both the base
and the exponent. It has two synonyms: Power\index{Power} and
XtoY\index{XtoY}.

\subsection{Hyperbolic Functions}

The AMath library has the same hyperbolic functions described in
\ref{sec:hyper}, but it adds inverse hyperbolics. The formulae
for these are:

$${\rm asinh} x = \ln\left(x + \sqrt{x^2+1}\right)$$
$${\rm acosh} x = \ln\left(x + \sqrt{x^2-1}\right), x \ge 1$$
$${\rm atanh} x = {1 \over 2}\ln\left({1 + x \over 1 - x}\right), -1 < x < 1$$

\begin{tabular}{lp{10cm}}
SinH\index{SinH}&{\it value} is a real number, and the function returns
its hyperbolic sine;\\
ASinH\index{ASinH}&{\it value} is a real number, and the function
returns its inverse hyberbolic sine;\\
CosH\index{CosH}&{\it value} is a real number, and the function returns
its hyperbolic cosine;\\
ACosH\index{ACosH}&{\it value} is a real number greater than or equal to
1, and the function returns its inverse hyperbolic cosine;\\
TanH\index{TanH}&{\it value} is a real number and the function returns
its hyperbolic tangent;\\
ATanH\index{ATanH}&{\it value} is a real number with absolute value less
than 1 and the function returns its inverse hyperbolic tangent.\\
\end{tabular}

\subsection{Numerical Functions}

The numerical functions are just miscellaneous functions, but calling
them numerical sounds more like they were thought out carefully. As with
almost every other function in this library, the calling sequence is:

\begin{quote}
Ceil({\it value}) {\rarrow} {\it result}
\end{quote}

\begin{tabular}{lp{10cm}}
Ceil\index{Ceil}&{\it value} is a real number and the function returns
the smallest integer greater than {\it value};\\
Floor\index{Floor}&{\it value} is a real number and the function returns
the largest integer greater than {\it value};\\
Int\index{Int}&Synonym for Floor;\\
NInt\index{NInt}&{\it value} is a real number and the function returns
nearest integer;\\
Fact\index{Fact}&{\it value} is an integer and the function returns
its factorial.
\end{tabular}

\clearpage % or 
\cleardoublepage

\phantomsection % fixes the link anchor

\addcontentsline{toc}{section}{
\indexname}


\printindex


\end{document}
